%%%Plantilla de ejercicios para la clase de Geometría Multivista. Dr. Bravo, Dra. Hevia, Dr. Pérez.
%%Semestre 2022-2 MCC UNAM
%%Elaborado por: Gibrán Zazueta
%https://github.com/gba2-gh/MViewGeometry_projects/tree/master/ejerciciosPlantilla

\documentclass[12pt]{article}

\usepackage{graphicx}
\usepackage{float}
\usepackage{amsmath}
\usepackage{amssymb}
\usepackage{graphicx}
\usepackage[utf8]{inputenc}
\usepackage[spanish]{babel}
\usepackage{geometry}
\geometry{left=2cm,right=2cm,top=2cm,bottom=2cm}
\usepackage{listings}
\lstset{basicstyle=\ttfamily,
  showstringspaces=false,
  commentstyle=\color{red},
  keywordstyle=\color{blue}
}


\title{%
Ejercicios Geometría proyectiva
}
\date{}

\begin{document}
\maketitle


\section{Ejercicio 1}
\textbf{
Utilice álgebra elemental (por ejemplo, Cramer) para determinar
que las rectas definidas por $r_{1}=[a_{1}, b_{1}, c_{1}]^{T}$ y $r_{2}=[a_{2},b_{2},c_{2}]^{T}$ intersecan en el punto $p=[b_{1}c_{2}-b_{2}c_{1}, a_{2}c_{1}-a_{1}c_{2}, a_{1}b_{2}-a_{2}b_{1}]^{T}$
}
\\

\section{Ejercicio 2}
\textbf{
Verifique que el punto de intersección $p$ entre las rectas $r_{1}= [a_{1}, b_{1}, c_{1}]^{T}$ y $r_{2}=[a_{2},b_{2},c_{2}]^{T}$ es de la forma $p=r_{1} \times r_{2}$, donde
“$\times$” denota el producto vectorial. (Nótese que la simplicidad de esta fórmula para calcular el punto de intersección de dos rectas es una consecuencia de la representación de rectas y puntos como vectores homogéneos).
}

\section{Ejercicio 3}
\textbf{
(Ejemplo 2.3, pág. 27 de Hartley y Zisserman)\\
El punto no homogéneo $p_{0}=[1, 1]^{T} \in \mathbb{R}^{2}$ es el punto de intersección de las
rectas $x=1$ e $y=1$ . Utilice la fórmula del Ejercicio 2 para encontrar la representación homogénea del punto $p_{0}$. .
}

\section{Ejercicio 4}
\textbf{
(Ejemplo 1, pág. 4 de Birchfield)\\
Demuestre que el punto de intersección de las rectas $r_{1}=[4, 2, 2]^{T}$ y $r_{2}=[6, 5, 1]^{T}$ es el punto $p=[-1, 1, 1]^{T}$
}

\section{Ejercicio 5}
\textbf{(Ecuación de la recta que pasa por dos puntos)\\
Demuestre que la ecuación de la recta r que pasa por los puntos $p_{1}=[x_{1}, y_{1}, z_{1}]^{T}$ y $p_{2}=[x_{2}, y_{2}, z_{2}]^{T}$ cumple que $r=p_{1} \times p_{2}$
}
\\


\section{Ejercicio 6}
\textbf{
(Colinealidad)\\
Demuestre que la condición para que los tres puntos $p_{1}=[x_{1}, y_{1}, z_{1}]^{T}$, $p_{2}=[x_{2}, y_{2}, z_{2}]^{T}$ y $p_{3}=[x_{3}, y_{3}, z_{3}]^{T}$ pertenezcan a la misma recta (o
que sean colineales) es $p_{3}^{T}(p_{1}\times p_{2})=0$, o sea que $det[p_{1}\ p_{2}\ p_{3}]=0$
}
\\


\section{Ejercicio 7}
\textbf{
(Concurrencia)\\
Demuestre que la condición para que las tres rectas $r_{1}=[a_{1}, b_{1}, c_{1}]^{T}$, $r_{2}=[a_{2}, b_{2}, c_{2}]^{T}$ y $r_{3}=[a_{3}, b_{3}, c_{3}]^{T}$ se intercepten en un mismo
punto (o sea, que sean concurrentes) es que $det[r_{1}\ r_{2}\ r_{3}]=0$
}
\\


\section{Ejercicio 8}
\textbf{
(Ejemplo 2.5, pág. 28 de Hartley y Zisserman)\\
Sean las rectas paralelas $x=1$ y $x=2$ del plano euclidiano $\mathbb{R}^{2}$, obtener el punto de intersección en el infinito en la dirección del eje y .
}
\\


\section{Ejercicio 9}
\textbf{Para 'homogenizar'(3), introducimos las siguientes sustituciones $x \rightarrow \frac{x_{1}}{x_{3}}$, $y \rightarrow \frac{x_{2}}{x_{3}}$}
\\

\textit{\textbf{
a) Comprobar que la correspondiente ecuación en coordenadas homogéneas es de la forma $$ax_{1}^2+bx_1x_2+cx_2^2+dx_1x_3+ex_{2}x_{3}+fx_3^2=0$$}}
\\


\textit{\textbf{
b) Demuestre que (4) es una ecuación polinómica homogénea de grado 2.}}
\\

\textit{\textbf{
c) Verificar que (4) se puede expresar en notación matricial por la ecuación $$x^{T}Cx = 0$$ donde $x=[x_{1}, x_{2}, x_{3}]^{T}$ y C es la matriz simétrica de los coeficientes de la cónica, definida por
$$C=\begin{bmatrix}
a & a/2 & d/2 \\
b/2 & c & e/2 \\
d/2 & e/2 & f
\end{bmatrix}$$
}}
\\


\section{Ejercicio 10}
\textbf{
(Resultado 2.7, pág. 31 de Hartley y Zisserman 21 )\\
 Demuestre que la ecuación de la recta tangente $r$ a la cónica $C$ en un punto $x$ está dada por $r=Cx$.
}
\\



\section{Ejercicio 11}

\textbf{(Cónicas degeneradas)\\
Estudiar concepto de cónica degenerada y el Ejemplo 2.8, pág. 32 de Hartley y Zisserman. Considerar el caso particular de las rectas 
$l= \begin{bmatrix} 1 & 0 & -1\end{bmatrix}^{T}, m= \begin{bmatrix} 0 & 1 & -1\end{bmatrix}^{T} $, y compruebe que:
}

\textbf{a) $C=lm^{T} + ml^{T} = \begin{bmatrix} 0 &1 & -1\\1 &0&-1\\-1&-1&2 \end{bmatrix}$}

\textbf{b) det[c] =0} 


\textbf{c)$ x^{T}Cx $ conduce a $ x_{1}x_{2}-x_{1}x_{3}-x_{2}x_{3}+x_{3}^{2}=0$ }

\textbf{d) La ecuación $ x_{1}x_{2}-x_{1}x_{3}-x_{2}x_{3}+x_{3}^{2}=0$ de $P^{2}$ transformada al plano euclidiano $\mathbb{R}^{2}$ es $(x-1)(y-1)=0$ }

\textbf{e) La solución del sistema Cx=0 es la recta $\alpha[1 1 1]^{T}, \alpha \in \mathbb{R}^{2}$}

\textbf{f) El punto de intersección entre las rectas es $l$ y $m$ es $[1 1 1]^{T}$}

\textbf{g) Remitirse al Ejercicio 3 para interpretar geométricamente los resultados
de los incisos anteriores.}

\section{Ejercicio 12}
\textbf{
Si los cuatro pares de puntos correspondientes, $(X_{i}, Y_{i}) \leftrightarrow (x_{i}, y_{i})$, para $i=1, 2, 3, 4$, satisfacen las ecuaciones (5), y suponiendo que $h_{33} =1$, demuestre que de (5) resulta el siguiente sistema de ecuaciones lineales...}


\section{Ejercicio 13}
\textbf{
Probar que las cónicas duales $R^{T}C^{*}R=0$ del plano proyectivo de partida se transforman proyectivamente en cónicas duales $r^{T}C^{*\prime}r =0$ del plano proyectivo de llegada donde $C^{*\prime}=HC^{*}H^{T}$ .
}


\section{Ejercicio 14}
\textbf{
Demuestre que la distancia Euclidiana entre dos puntos $P_{i}=[X_{i}, Y_{i}, Z_{i}]^{T}$ y $P_{j}=[X_{j}, Y_{j}, Z_{j}]^{T}$  (para i, j =1,2,3,4) calculada a partir de sus correspondientes puntos del plano Euclidiano  $\mathbb{R}^{2}$ se
expresa por la fórmula $\Delta_{ij} = \sqrt{ \left(\frac{X_{i}}{Z_{i}} - \frac{Y_{i}}{Z_{i}}\right)^{2} +  \left(\frac{X_{j}}{Z_{j}}, \frac{Y_{j}}{Z_{j}} \right)^{2}} $
}
\\


\section{Ejercicio 15}
\textbf{
Escriba cinco posibles fórmulas para razón cruzada de cuatro puntos colineales $P_{1},P_{2},P_{3},P_{4}$ de $P^{2}$ , ¿Cuántas fórmulas posibles pudieran encontrarse?
}
\\


\section{Ejercicio 16}


\textbf{
Demuestre que la transformación proyectiva sobre la recta R es de la forma $x=\frac{h_{11}X+h_{12}}{h_{21}X+h_{22}}$, la cual es la forma bidimensional de la ecuación (5).
}
\\



\section{Ejercicio 17}
\textbf{
Demuestre que la distancia Euclidiana entre dos puntos $P_{i}=[X_{i}, Y_{i}]^{T}$ y $P_{j}=[X_{j}, Y_{j}]^{T}$ de $P^{1}$ (para i, j =1,2,3,4) calculada a partir de sus correspondientes puntos de la recta Euclidiana $\mathbb{R}$ se calcula por la fórmula $\Delta_{ij} =  \frac{1}{|Y_{i}Y{j}|} det[P{i}P{j}]$
}
\\



\section{Ejercicio 18}

\textbf{
Demuestre que 
$Cr(P_{1},P_{2},P_{3},P_{4})= \frac{ det[P_{1}P_{3}]det[P_{2}P_{4}] }{ det[P_{2}P_{3}]det[P_{1}P_{4}]   }$
}
\\


\section{Ejercicio 19}
\textbf{
Demuestre que bajo la transformación proyectiva $p =HP$ se cumple que $Cr(p1,p2,p3,p4) = Cr(P1,P2,P3,P4)$
}
\\


\section{Ejercicio 20}
\textbf{
Demuestre que una transformación es una transformación de semejanza, sí y solo sí, esta
mantiene invariantes los puntos absolutos $i=[1,i,0]^{T}$ y $j=[1,-i,0]^{T}$
}
\\



\section{Ejercicio 21}
\textbf{(El objetivo de este ejercicio es obtener una fórmula clásica para determinar el ángulo entre las correspondientes proyecciones de las rectas $r_{1}$ y $r_{2}$ en el plano Euclidiano)
\\
Dadas las rectas  $r_{1}=[a_{1}, -1, 0]^{T}$ y $r_{2}=[a_{2}, -1, 0]^{T}$ del plano proyectivo $P^{2}$:
}
\\

\textbf{
a)Verifique que sus correspondientes ecuaciones cartesianas (en el plano Euclidiano $R^{2}$) son  $a_{1}x -y =0$ y  $a_{2}x -y =0$, respectivamente.
}
\\


\textbf{
b)Compruebe que sus ecuaciones vectoriales son, respectivamente, $r_{1}(\alpha)=\alpha[1, a_{1}]^{T}$ y $r_{2}(\alpha)=\alpha[1, a_{2}]^{T}$
}


\textbf{
c)El ángulo $\theta$ entre las rectas $r_{1}$ y $r_{2}$ coincide con el ángulo entre los
vectores directores $v_{1} = [1. a_{1}]$ y $v_{2} = [1. a_{2}]$. Demuestre que
$$tan\theta = \frac{|v_{2} \times v_{1} |}{v_{2} \cdot v_{1}} = \frac{a_{1} - a_{2}}{1+a_{1}a_{2}}$$
}


\section{Ejercicio 22 }

\textbf{El objetivo de este ejercicio es obtener una fórmula clásica para
determinar el ángulo entre las rectas $r_{1}$ y $´r_{2}$ en el plano proyectivo y comparar
con la obtenida en el Ejercicio 21)}
\\

Dadas las rectas  $r_{1}=[a_{1}, -1, 0]^{T}$ y $r_{2}=[a_{2}, -1, 0]^{T}$ del plano proyectivo $P^{2}$:
\\

\textit{\textbf{
a) Comprobar que los puntos de intersección $p_{i}$ entre las rectas $r_{i}$ y la
recta ideal $r_{\infty}$ son de la forma $p_{i}=[1, a_{i},0]^{T}$ $(i=1,2)$.
}}
\\



\textit{\textbf{b) Comprobar que la razón cruzada entre los puntos $p_{1}=[1,a_{1},0]^{T}, p_{1}=[1,a_{2},0]^{T}, i=[1,i,0]^{T}$ y $j=[1,-i,0]^{T}$
está dada por $$Cr(p_{1},p_{2},i,j) =e^{2i\left(tan^{-1}\left(\frac{a_{1}-a_{2}}{a_{1}a_{2} + 1}\right)\right)}$$
}}
\\


\textit{\textbf{c)Comprobar de (26) y (27) resulta (25).}}
\\


\section{Ejercicio 23}
\textbf{(El objetivo de este ejercicio es visualizar la interpretación geométrica Euclidiana de la recta r que pasa por dos puntos no ideales $P_{1}$ y $P_{2}$ de $P^{3}$)\\
Sea r la recta de $P^{3}$ que pasa por los puntos $P_{1} = [1, 1, 0,1]^{T}$ y  $P_{2} = [0, 1, 1 ,1]^{T}$}
\\

\textit{\textbf{
a) Escribir la ecuación de la recta r usando la fórmula (10).}}
\\



\textit{\textbf{
b) Hallar los puntos $\overline{P}_{1}$ y $\overline{P}_{2}$ , correspondientes de $\mathbb{R}^{3}= \{(x,y,z)\}$, en coordenadas no homogéneas.}}
\\



\textit{\textbf{
c) Hallar la ecuación vectorial de la recta $\overline{r}$ que pasa por los puntos $\overline{P}_{1}$ y $\overline{P}_{2}$}}
\\


\textit{\textbf{
d)Hallar la ecuación cartesiana $Ax+By+Cz=0$ del plano que pasa por el origen y por los puntos $\overline{P}_{1}$ y $\overline{P}_{2}$}}
\\

\textit{\textbf{
e) Ilustrar gráficamente en el primer octante del espacio Cartesiano $R^{3} ={(x,y,z)}$, a los puntos $\overline{P}_{1}$ y $\overline{P}_{2}$, al plano que pasa por el origen y por los puntos $\overline{P}_{1}$ y $\overline{P}_{2}$ , y al vector normal a dicho plano $n= \overline{P}_{1} \times \overline{P}_{2}$}}




\section{Ejercicio 24}
\textbf{
(Sucesión de pasos que conduce a la ecuación del plano que pasa por tres puntos de $P^{3}$ a partir del vector normal en coordenadas no homogéneas)}
\\

\textit{\textbf{
a) Verificar que, si $D\neq 0$ , el sistema (13) se puede expresar en notación
matricial como sigue: 
}}
$$ 
\begin{bmatrix}
x_{1} & y_{1}  & z_{1} \\
x_{2} & y_{2}  & z_{2} \\
x_{3} & y_{3}  & z_{3} \\
\end{bmatrix}
\begin{bmatrix}
A/D \\
B/D \\
C/D  \\
\end{bmatrix}  =
\begin{bmatrix}
-1 \\
-1 \\
-1  \\
\end{bmatrix}  
 $$
\\`


\textit{\textbf{
b) Suponiendo que $\begin{vmatrix}
x_{1} & y_{1}  & z_{1} \\
x_{2} & y_{2}  & z_{2} \\
x_{3} & y_{3}  & z_{3} \\
\end{vmatrix} \neq 0$,
aplicar Cramer, para comprobar que
las componentes $\bar{n_{x}} \bar{n_{y}} \bar{n_{z}}$  del vector normal $\bar{n}=[\bar{n_{x}} \bar{n_{y}} \bar{n_{z}}]^{T}$
satisfacen las siguientes relaciones:
}}

$$n_{x}= \frac{\begin{vmatrix}
-1 & y_{1}  & z_{1} \\
-1 & y_{2}  & z_{2} \\
-1 & y_{3}  & z_{3} 
\end{vmatrix} }{\begin{vmatrix}
x_{1} & y_{1}  & z_{1} \\
x_{2} & y_{2}  & z_{2} \\
x_{3} & y_{3}  & z_{3} 
\end{vmatrix}},
n_{y}  = \frac{\begin{vmatrix}
x_{1} & -1  & z_{1} \\
x_{2} & -1  & z_{2} \\
x_{3} & -1  & z_{3} 
\end{vmatrix} }{\begin{vmatrix}
x_{1} & y_{1}  & z_{1} \\
x_{2} & y_{2}  & z_{2} \\
x_{3} & y_{3}  & z_{3} 
\end{vmatrix}},
n_{z}  = \frac{\begin{vmatrix}
x_{1} & y_{1}  & -1 \\
x_{2} & y_{2}  & -1 \\
x_{3} & y_{3}  & -1  
\end{vmatrix} }{\begin{vmatrix}
x_{1} & y_{1}  & z_{1} \\
x_{2} & y_{2}  & z_{2} \\
x_{3} & y_{3}  & z_{3} 
\end{vmatrix}}$$
\\

\textit{\textbf{
c)Homogeneizar el vector normal $\bar{n}=[\bar{n_{x}} \bar{n_{y}} \bar{n_{z}}]^{T}$ mediante los cambios $x_{i}=\frac{X_{i}}{W_{i}}$, $y_{i}=\frac{Y_{i}}{W_{i}}$  y $z_{i}=\frac{Z_{i}}{W_{i}}$ aplicando propiedades de los determinantes, obtener la siguiente ecuación del plano que pasa por los puntos $P_{1} = [X_{1},Y_{1}, Z_{1}]^{T}$, $P_{2} = [X_{2},Y_{2}, Z_{2}]^{T}$ y $P_{3} = [X_{3},Y_{3}, Z_{3}]^{T}$ de $P^{3}$
}}
$$\bar{n} = \left[ \begin{vmatrix}
Y_{1} & Y_{2}  & Y_{3} \\
W_{1} & W_{2}  & W_{3} \\
Z_{1} & Z_{2}  & Z_{3} \\
\end{vmatrix}  ,
\begin{vmatrix}
X_{1} & X_{2}  & X_{3} \\
Z_{1} & Z_{2}  & Z_{3} \\
W_{1} & W_{2}  & W_{3} \\
\end{vmatrix} ,
\begin{vmatrix}
X_{1} & X_{2}  & X_{3} \\
W_{1} & W_{2}  & W_{3} \\
Y_{1} & Y_{2}  & Y_{3} \\
\end{vmatrix} ,
\begin{vmatrix}
X_{1} & X_{2}  & X_{3} \\
Y_{1} & Y_{2}  & Y_{3} \\
Z_{1} & Z_{2}  & Z_{3} \\
\end{vmatrix}
\right]^{T}$$
\\

\section{Ejercicio 25}
\textbf{
(Tres planos concurrentes en un punto)
Deduzca, por dualidad con (16), la fórmula del punto de intersección $p$ entre los
planos $n_{1}=[n_{1}^{1}, n_{2}^{1}, n_{3}^{1}, n_{4} ^{1}]$, $n_{2}=[n_{1}^{2}, n_{2}^{2}, n_{3}^{2}, n_{4} ^{2}]$ y $n_{3}=[n_{1}^{3}, n_{2}^{3}, n_{3}^{3}, n_{4} ^{3}]$ de
$P^{3}$ .
}


\end{document}



